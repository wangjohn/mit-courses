\documentclass[12pt]{article}

\usepackage{geometry}
\usepackage{fancyhdr}
\usepackage{amsmath}
\usepackage{graphicx}
\usepackage{multicol}
\usepackage{setspace}
\usepackage{mathrsfs}


\hoffset -0.4in
\rhead{ }
\headheight -18pt
\footskip 45pt
\textheight 620pt
\textwidth 500pt
\marginparwidth = -30pt

\begin{document}

\begin{center}
\Huge Rotational Kinematics with Harmonic Motion \\
\vspace{0.1cm}
\hline
\vspace{0.1cm}
\large Irene Chen, Aaron Downward, and John Wang \\ 
\normalsize 8.012 Project Proposal\\

\end{center}

\section{Description of Project}

\paragraph{} This project will build a pendulum apparatus and analyze its motion. This pendulum will consist of two parts. First, a vertical rod will hang from a circular ring that acts as the pivot point. Second, a rubber band will attach the vertical rod to a horizontal bar with two paint brushes fastened to the ends (see attached page for diagram). This entire device will have two regimes of motion. One regime will be the harmonic, pendulum-like motion of the entire object about the circular ring at the top. The other regime will result from the rotation of the horizontal bar about its center of mass. This project will determine the picture that the paintbrushes trace on the ground. This will be done analytically by examining the physics of the two regimes and we shall verify our results with experimental evidence from our apparatus. 

\section{Description of Physical Concepts}

\paragraph{} Our project incorporates harmonic motion and basic forces such as tension and friction. The path of the rod attached to the circular ring mimics that of a pendulum, and in order to sufficiently analyze how altering certain parameters will change the motion of the entire system, mastery of pendulums and simple harmonic motion is required. The second part of our apparatus necessitates the understanding of tension, the use of polar coordinates to simplify calculations involving centripetal acceleration, and the incorporation of angular momentum and moment of inertia. Finally, a firm grasp of how friction impacts motion is indispensable in determining how the path of the pendulum rod is affected by the friction between the brushes and the sheet of paper.

\section{Description of Materials}

\paragraph{} To construct the pendulum and the horizontal bar, we will use K'NEX, a toy construction system consisting of a rods and connectors. This will be an effective material because the parts are light, have a low coefficient of friction, and are easy to adjust based on our needs. We will order these inexpensive K'NEX toys through the internet. For the paintbrushes, we will simply browse local stores to find a pair of brushes that will suit our needs. Similarly, it will be easy to move the center of mass to the bottom our pendulum with a small bag of rocks or another source of weight as we see fit. To set the horizontal bar into motion, we are planning to use a large rubber band, which can be obtained in a hardware or office supply store. However, we are also looking into using multiple rubber bands and will likely experiment with other sources to provide a more efficient motion. Apart from the main pendulum, we will also need paint and a large sheet of paper, both of which can also be obtained at a hardware store.

\end{document}



