\documentclass[psamsfonts]{amsart}

%-------Packages---------
\usepackage{amssymb,amsfonts}
\usepackage[all,arc]{xy}
\usepackage{enumerate}
\usepackage[margin=1in]{geometry}
\usepackage{amsthm}
\usepackage{theorem}
\usepackage{verbatim}
\usepackage{tikz}
\usetikzlibrary{shapes,arrows}

\newenvironment{sol}{{\bfseries Solution:}}{\qedsymbol}
\newenvironment{prob}{{\bfseries Problem:}}

\bibliographystyle{plain}

\voffset = -10pt
\headheight = 0pt
\topmargin = -20pt
\textheight = 690pt

%--------Meta Data: Fill in your info------
\title{Knewton Data Challenge}
\author{John Wang}

\begin{document}

\maketitle

\section{Problem Formulation}

Let there be $n$ individuals taking an exam each year, each of whom take $k$ of the $K$ total questions available. We require that $L$ of the $K$ questions be offered to at least one student, and also that $0 < k < K$ be satisfied. In this paper, I will examine strategies for finding a ranking of students given a previous year's results as training data.

\section{Question Difficulty}

In order to do this, we want to formulate some measure of the difficulty of each question $j$. This motivates our examination of $r_j$, the probability that a student will get question $j$ correct. To estimate $r_j$, one could naively use the sample mean from the training data for each question:
\begin{eqnarray}
r_j \approx \frac{1}{n_j} \sum_{i=1}^n x_{ij}
\end{eqnarray}

Where $x_{ij}$ denotes whether or not individual $i$ answered question $j$ correctly and $n_j$ is the number of times question $j$ was asked in the training data.


\end{document}
