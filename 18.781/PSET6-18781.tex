\documentclass[psamsfonts]{amsart}

%-------Packages---------
\usepackage{amssymb,amsfonts}
\usepackage[all,arc]{xy}
\usepackage{enumerate}
\usepackage{mathrsfs}
\usepackage[margin=1in]{geometry}
\usepackage{thmtools}
\usepackage{verbatim}
\usepackage{multirow}


%--------Theorem Environments--------
%theoremstyle{plain} --- default
\newtheorem{prob}{Problem}[section]
\newtheorem{thm}{Theorem}[section]
\newtheorem{cor}[thm]{Corollary}
\newtheorem{prop}[thm]{Proposition}
\newtheorem{lem}[thm]{Lemma}
\newtheorem{conj}[thm]{Conjecture}
\newtheorem{quest}[thm]{Question}

\newenvironment{sol}{{\bfseries Solution}}{\qedsymbol}

\def\legendre(#1,#2){%
{#1 \overwithdelims () #2} }

\theoremstyle{definition}
\newtheorem{defn}[thm]{Definition}
\newtheorem{defns}[thm]{Definitions}
\newtheorem{con}[thm]{Construction}
\newtheorem{exmp}[thm]{Example}
\newtheorem{exmps}[thm]{Examples}
\newtheorem{notn}[thm]{Notation}
\newtheorem{notns}[thm]{Notations}
\newtheorem{addm}[thm]{Addendum}
\newtheorem{exer}[thm]{Exercise}

\theoremstyle{remark}
\newtheorem{rem}[thm]{Remark}
\newtheorem{rems}[thm]{Remarks}
\newtheorem{warn}[thm]{Warning}
\newtheorem{sch}[thm]{Scholium}


\makeatletter
\let\c@equation\c@thm
\makeatother
\numberwithin{equation}{section}

\bibliographystyle{plain}

\voffset = -10pt
\headheight = 0pt
\topmargin = -20pt
\textheight = 690pt

%--------Meta Data: Fill in your info------
\title{18.781 \\
Problem Set 6}

\author{John Wang}

\begin{document}

\maketitle

\section{Problem 1}

\begin{prob}
Show that $\prod_{d|n} d = n^{d(n)/2}$. 
\end{prob}

\begin{sol}
We note that if $d$ is a divisor of $n$, then $\frac{n}{d}$ must also be a divisor of $n$ which is distinct from $d$, since clearly $d | n$ by definition. The divisor must be distinct unless $n$ is a square, and $d^2 = n$, but we shall still count those as two distinct divisors. 

Since each divisor $d$ must also have a pair $\frac{n}{d}$, and we know that the product of the pair is $n$, we can match up all of the divisors of $n$ in the following way, letting $d_i$ be the first divisor in the pair: 
\begin{eqnarray}
\prod_{d|n} d &=& \prod_{i = 1}^{d(n)/2} d_i \frac{n}{d_i} \\
&=& \prod_{i=1}^{d(n)/2} n \\
&=& n^{d(n)/2}
\end{eqnarray}
\end{sol}

\section{Problem 2}

\begin{prob}
If $k$ is a positive integer, show that $\sigma_k(n)$ is odd if and only if $n$ is a square or twice a square.
\end{prob}

\begin{sol}
First, we note that $\sigma_k(n)$ is a multiplicative function so that $\sigma_k(ab) = \sigma_k(a) \sigma_k(b)$ as shown in class. Also, we know that $\sigma_k(p^e) = 1 + p^k + p^{2k} + \ldots + p^{ek}$ because the only divisors of $p^e$ are $1, p, p^2, \ldots, p^e$. Summing up the $k$th powers gives the expression. Therefore, we can decompose $\sigma_k(n)$ for any arbitrary integer $n$ using the prime factorization theorem, where $n = 2^{e_0} p_1^{e_1} p_2^{e_2} \ldots p_r^{e_r}$:
\begin{eqnarray}
\sigma_k(n) &=& \sigma_k(2^{e_0}) \sigma_k(p_1^{e_1}) \sigma_k(p_2^{e_2}) \ldots \sigma_k(p_r^{e_r}) \\
&=& (1 + 2 + 2^2 + \ldots + 2^{e_0}) ( 1 + p_1^k + p_1^{2k} + \ldots + p_1^{e_1 k} ) \ldots \left( 1 + p_r^k + p_r^{2k} + \ldots + p_r^{e_r k} \right)
\end{eqnarray}

Now, we see that $(1 + 2 + 2^2 + \ldots 2^{e_0})$ must always be odd, because $2^i$ for $i \geq 1$ is always even, so that $\sum_{i=1}^{e_0} 2^i$ is also even. This means that $1 + \sum_{i=1}^{e_0} 2^i = 1 + 2 + 2^2 + \ldots 2^{e_0} $ must be odd. Now, we invoke the following lemma:

\begin{lem}
Let $p$ be an odd prime. The sum $1 + p^k + p^{2k} + \ldots + p^{e k}$ is odd if and only if $e$ is even. 
\end{lem}

\begin{proof}
We know that any odd integer multiplied by another odd integer results in an odd integer. Thus, $p^k$ must be odd because $p$ is an odd prime. This means that each term in the sequence $\{p^k, p^{2k}, \ldots, p^{e k} \}$ must be odd because they are just $p^k$ multiplied by more odd numbers. 

Moreover, we know that a sum of odd numbers is odd if and only if there are an odd number of them. This implies that there must be an odd number of terms in $1 + p^k + p^{2k} + \ldots + p^{e k}$ in order for it to be odd. This occurs if and only if $e$ is even.
\end{proof}

Now, if all of the $e_i$ are even, we can rewrite $e_1, e_2, \ldots, e_r$ in the form $k_i = e_i / 2$, and set:
\begin{equation}
k_0 = \left\{ \begin{array}{c l}
\frac{e_0}{2} & \textrm{if } e_0 \textrm{ is even} \\
\frac{e_0 - 1}{2} & \textrm{if } e_0 \textrm{ is odd} 
\end{array} \right.
\end{equation}

Notice that we can do this if and only if $n$ is a square because $n = (2^{k_0} p_1^{k_1} \ldots p_r^{e_k})^2$ if $e_0$ is even or $n = 2 (2^{k_0} p_1^{k_1} \ldots p_r^{e_k})^2$ if $e_0$ is odd. However, if all $e_i$ are even the, by the lemma shown above, each term $\sigma_k(p_i^{e_i})$ must be odd, so the product of all these terms must also be odd. Notice that this happens if and only if $n$ is a square in one of the above two forms. 
\end{sol}

\section{Problem 3}

\begin{prob}
Prove that if $(a,b) > 1$ then $\sigma_k(ab) < \sigma_k(a) \sigma_k(b)$ and $d(ab) < d(a)d(b)$. 
\end{prob}

\begin{sol}
First, we note that $\sigma_k(n) = \sum_{d|n} d^k$ by definition. However, we can decompose $n$ into is prime factors $n = p_1^{e_1} p_2^{e_2} \ldots p_r^{e_r}$. Now note that all divisors of $n$ must be composed of these prime powers $\prod_{i=1}^{r} p_i^{a_i}$ where $a_i \in \{0, 1, \ldots, e_i\}$. Moreover, we note that every permutation of $a_i \in \{0, 1, \ldots e_i\}$ for all $i$ must occur to create all the distinct factors of $n$. Therefore, we can write:
\begin{eqnarray}
\sigma_k(n) &=& \left(1 + p_1^k + p_1^{2k} + \ldots + p_1^{e_1 k} \right) \left(1 + p_2^k + p_2^{2k} + \ldots + p_2^{e_2 k} \right) \ldots  \left(1 + p_2^k + p_2^{2k} + \ldots + p_2^{e_2 k} \right) \\ 
&=& \prod_{i=1}^{\omega(n)} \left(1 + p_i^k + p_i^{2k} + \ldots + p_i^{e_i k} \right)
\end{eqnarray} 

Where $\omega(n)$ is the number of distinct primes dividing $n$. Note that this particular way of writing $\sigma_k(n)$ comes about because the above multiplication runs over all the permutations of $a_i \in \{0,1, \ldots, e_i\}$ for all $i$ and therefore covers all of 
the divisors of $n$. 

Now consider two integers $a,b$ such that $(a,b) > 1$. This implies that $a$ and $b$ share at least one prime factor $p_i$. This means one or more terms $p_i^{a_i}$ occur in both $a$ and $b$. However, these terms $p_i^{a_i}$ occur only once in the expansion of $ab$. Now let there be $q$ prime factors that are repeated, each with $a_{jk} \in \{0,1, \ldots, e_j\}$. We have the following:
\begin{eqnarray}
\sigma_k(ab) &=& \prod_{i=1}^{\omega(ab)}  \left(1 + p_i^k + p_i^{2k} + \ldots + p_i^{e_i k} \right) \\
&=&  \left( \prod_{i=1}^{\omega(a)} \left(1 + p_i^k + p_i^{2k} + \ldots + p_i^{e_i k} \right) \right) \left( \prod_{i=1}^{\omega(b)} \left(1 + p_i^k + p_i^{2k} + \ldots + p_i^{e_i k} \right) \right) - \sum_{j=1}^q \sum_{k=0}^{e_j} p_j^{a_{jk}} \\
&=& \sigma_k(a) \sigma_k(b)  - \sum_{j=1}^q \sum_{k=0}^{e_j} p_j^{a_{jk}}
\end{eqnarray}

Since $\sum_{j=1}^q \sum_{k=0}^{e_j} p_j^{a_{jk}} > 0$ because there are repeated factors, we know that $\sigma_k(ab) < \sigma_k(a) \sigma_k(b)$. Moreover, since $d(n) = \sigma_0(n)$ is just a special case of $\sigma_k(n)$ where $k = 0$, we have also shown that $d(ab) < d(a) d(b)$ when $(a,b) > 1$. 
\end{sol}

\section{Problem 4}

Recall that a perfect number $n$ is one for which $\sigma(n) = 2n$. 

\begin{prob}
If $p = 2^{m} - 1$ is a prime, show that $2^{m-1} (2^{m} - 1)$ is a perfect number.
\end{prob}

\begin{sol}
We must show that $\sigma(2^{m-1}(2^{m} - 1)) = 2^m (2^m - 1)$. Since we know that $\sigma(n)$ is multiplicative, and we know that $(2^{m-1}, 2^{m} - 1) = 1$ because $2^{m} - 1$ is a prime, we find that:
\begin{eqnarray}
\sigma(2^{m-1}(2^{m} - 1)) &=& \sigma(2^{m-1}) \sigma(2^{m} - 1) \\
&=& \sigma(2^{m-1}) (1 + 2^{m} - 1) \\
&=& \sigma(2^{m-1}) 2^m
\end{eqnarray}

This follows since the only divisors of $2^{m} - 1$ are itself and 1. Next, we must compute $\sigma(2^{m-1})$. However, we know that the only divisors are powers of two. Thus, we have:
\begin{eqnarray}
\sigma(2^{m-1}) &=& 1 + 2 + 2^2 + \ldots + 2^{m-1} \\
&=& \sum_{i=0}^{m-1} 2^i \\
&=& \frac{2^{m-1 + 1} - 1}{2 - 1} \\
&=& 2^{m} - 1
\end{eqnarray}

Multiplying these together, we find that $\sigma(2^{m-1} (2^{m} -  1)) = 2^m (2^{m} - 1)$, which is exactly what we wanted to show.
\end{sol}

\begin{prob}
Show that every even perfect number is of the form $2^{m-1} (2^{m} - 1)$. 
\end{prob}

\begin{sol}
First, if $n$ is an even number, then we can decompose it into $n = 2^r s$ where $s$ is odd. We know that $\sigma(n) = \sigma(2^{r}) \sigma(s)$ by the multiplicative property and since $(2^{r}, s) = 1$. Moreover, we know that $\sigma(2^{r}) = \sum_{i=0}^r 2^i = 2^{r+1} - 1$. Therefore, we see that $\sigma(n) = (2^{r+1} - 1) \sigma(s)$. However, since $n$ is also a perfect number, we know that $\sigma(n) = 2 (2^r s) = 2^{r+ 1} s$. We can equate these two expressions for $\sigma(n)$ and obtain:
\begin{eqnarray}
(2^{r+1} - 1) \sigma(s) = 2^{r+1} s
\end{eqnarray}

This shows that $2^{r+1} - 1 | 2^{r+1} s$. Since it is clear that $2^{r+1} - 1 \nmid 2^{r+1}$, we must have $2^{r+1} - 1| s$. This implies that we can rewrite $s$ in the form $s = k (2^{r+1} - 1)$. Substituting this back into our expression, we find:
\begin{eqnarray}
(2^{r+1} - 1) \sigma(s) &=& 2^{r+1} k (2^{r+1} - 1) \\
\sigma(s) &=& 2^{r+1} k
\end{eqnarray}

However, we see that $s = k (2^{r+1} - 1) = k 2^{r+1} - k$. This shows that $2^{r+1} k = s + k$. Thus, we have found that $\sigma(s) = s + k$. Since $\sigma(\cdot)$ is the sum of divisors function, this can only occur when $s$ is prime and $k = 1$. This means that $s = 2^{r+1} - 1$ is a Marsenne prime. This means that $n = 2^{r} (2^{r+1} - 1)$ is a perfect number with the form that we wanted to show.   
\end{sol}

\section{Problem 5}

\begin{prob}
For any positive integer $n$, let $\lambda(n) = (-1)^{\Omega(n)}$. This is the Liouville's lambda function. Show that $\lambda(n)$ is totally multiplicative and that
\begin{equation}
\sum_{d|n} \lambda(d) = \left\{ \begin{array}{ l l}
1 & n \textrm{ is a perfect square} \\
0 & \textrm{otherwise}
\end{array} \right.
\end{equation}
\end{prob}

\begin{sol}
First, we shall show that $\lambda(n)$ is completely multiplicative. We know by the Fundamental Theorem of Arithmetic that $n = p_1^{e_1} \ldots p_r^{e_r}$ is a unique factorization. Therefore, we know that $\Omega(n)$, the number of primes dividing $n$ with multiplicity, is just $e_1 + e_2 + \ldots + e_r$. Therefore, $\Omega(ab) = \Omega(a) + \Omega(b)$ since we can decompose $a$ and $b$ to be $a = \prod_{i=1}^{r} p_i^{e_{a_i}}$ and $b = \prod_{i=1}^{r} p_i^{e_{b_i}}$ such that $a_i + b_i = e_i$. This means $\Omega(a) = a_1 + a_2 + \ldots + a_r$ and $\Omega(b) = b_1 + b_2 + \ldots + b_r$, which shows that $\Omega(ab) = e_1 + e_2 + \ldots + e_r$, which is what we wanted. Since $\Omega(n)$ is completely additive, we see that $\lambda(ab) = (-1)^{\Omega(ab)} = (-1)^{\Omega(a) + \Omega(b)} = (-1)^{\Omega(a)} (-1)^{\Omega(b)} = \lambda(a) \lambda(b)$. 

Now, we shall prove the second part of the problem. First, we can decompose any number into $n = p_1^{e_1} p_2^{e_2} \ldots p_r^{e_r}$. Now, summing over all divisors of $n$ consistitutes summing over all permutations of $p_1^{a_1} p_2^{a_2} \ldots p_r^{a_r}$ where $a_i \in \{0, 2, \ldots, e_i \}$. Thus, we can write the sum as:
\begin{eqnarray}
\sum_{d|n} \lambda(d) &=& \prod_{i=1}^{r} \left( \lambda(1) + \lambda(p_i) + \lambda(p_i^2) + \ldots + \lambda(p_i^{e_i}) \right) \\
&=& \prod_{i=1}^{r} \left( \lambda(1) + \lambda(p_i) + \lambda(p_i)^2 + \ldots + \lambda(p_i)^{e_i} \right)
\end{eqnarray}

Now notice that $\lambda(1) = (-1)^{\Omega(1)} = (-1)^{0} = 1$ since there are no prime divisors of $1$. Moreover, there is exactly one prime divisor for every prime $p_i$, namely $p_i$. Thus, for a prime $p_i$, we must have $\lambda(p_i) = (-1)^{1} = -1$. Knowing this, we can rewrite the above expression to:
\begin{eqnarray}
\sum_{d|n} \lambda(d) &=& \prod_{i=1}^{r} \left( \lambda(1) + \lambda(p_i) + \lambda(p_i)^2 + \ldots + \lambda(p_i)^{e_i} \right) \\
&=& \prod_{i=1}^{r} 1 + (-1) + (-1)^2 + (- 1)^3 + \ldots + (-1)^{e_i} \\ 
&=& \prod_{i=1}^{r} 1 + (-1)^{e_i}
\end{eqnarray}

Thus, if $e_i$ is even, then the series $(-1) + (-1)^2 + \ldots + (-1)^{e_i}$ telescopes to 0. If $e_i$ is odd, the series telescopes to $-1$. Thus, we find:
\begin{equation}
\sum_{d|n} \lambda(d) = \prod_{i=1}^r f(e_i) \hspace{0.5cm} \textrm{ where } f(e_i) = \left\{ \begin{array}{l l}
0 & \textrm{if } e_i \textrm{ is odd} \\
1 & \textrm{if } e_i \textrm{ is even} 
\end{array} \right.
\end{equation}

Now we note that all $e_i$ are even if and only if $n$ is a perfect square. This follows because if $e_i$ are all even, one can write $n$ as $n = (p_1^{e_1/2} p_2 ^{e_2 / 2} \ldots p_r^{e_r / 2})^2$, which shows that $n$ is a perfect square. If $n$ is a perfect square, then each $e_i$ must be divisible by 2 because $n = (p_1^{a_1} p_2^{a_2} \ldots p_r^{a_r})^2$, so each $e_i = 2 a_i$. Since $\sum_{d|n} \lambda(d) = 1$ if and only if $f(e_i) = 1$ for all $i$ (which occurs if and only if all $e_i$ are even), we see that the statement we wanted to prove is correct. 
\end{sol}

\section{Problem 6}

\begin{prob}
Show that 
\begin{equation}
\sum_{d|n} d(d)^3 = \left( \sum_{d|n} d(d) \right)^2
\end{equation}
for all positive integers $n$. 
\end{prob}

\begin{sol}
Let $n = p_1^{e_1} p_2^{e_2} \ldots p_r^{e_r}$ by the Fundamental Theorem of Arithmetic. First, we know that $d(n)$ is multiplicative so that $d(ab) = d(a) d(b)$ when $(a,b) = 1$. Namely, this means that we can expand $\sum_{d|n} d(d)$ and $\sum_{d|n} d(d)^3$ into a product of sums. We find:
\begin{eqnarray}
\sum_{d|n} d(d)^3 &=& \prod_{i=1}^r d(1)^3 + d(p_i)^3 + d(p_i^2)^3 + \ldots + d(p_i^{e_i})^3 \\
&=& \prod_{i=1}^r 1^3 + (1+ 1)^3 + (2 + 1)^3 + \ldots + (e_i + 1)^3 \\
&=& \prod_{i=1}^r \sum_{j=0}^{e_i} (j + 1)^3 \\
&=& \prod_{i=1}^r \frac{1}{4} (e_i + 1)^2 (e_i + 2)^2  
\end{eqnarray}

Where we have used the fact that $d(p^e) = e+1$ if $p$ is a prime, because all of the factors of $p^e$ are $\{1, p, p^2, \ldots, p^e\}$ which contains $e+1$ elements. We have also used the fact that $(d(ab))^3 = (d(a) d(b))^3 = d(a)^3 d(b)^3$ because $d$ is multiplicative. Next, we shall expand out the right hand side:
\begin{eqnarray}
\left( \sum_{d|n} d(d) \right)^2 &=& \left( \prod_{i=1}^r d(1) + d(p_i) + d(p_i^2) + \ldots + d(p_i^{e_i}) \right)^2 \\ 
&=& \left( \prod_{i=1}^r 1 + (1 +1) + (2 + 1) + \ldots + (e_i + 1) \right)^2 \\
&=& \left( \prod_{i=1}^r \sum_{j=0}^{e_i} (j+1) \right)^2 \\
&=& \left( \prod_{i=1}^r \frac{1}{2} (e_i + 1)(e_i + 2) \right)^2 \\
&=& \prod_{i=1}^r \frac{1}{4} (e_i + 1)^2 (e_i + 2)^2 
\end{eqnarray}

Clearly, these are the same, so we have finished the proof.
\end{sol}

\section{Problem 7}

\begin{prob}
Suppose $f(n)$ is an arithmetic function whose values are all nonzero, and put $\hat{F}(n) = \prod_{d|n} f(d)$. Show that
\begin{equation}
f(n) = \prod_{d|n} \hat{F}(d)^{\mu(n/d)}.
\end{equation}
\end{prob}

\begin{sol}
We shall expand out the right hand side, noting that since we range over $d|n$, we can exchange $d$ for $n/d$:
\begin{eqnarray}
\prod_{d|n} \hat{F}(d)^{\mu(n/d)} &=& \prod_{d|n} \hat{F}(n/d)^{\mu(d)} \\
&=& \prod_{d|n} \left( \prod_{e | \frac{n}{d}} f(e) \right)^{\mu(d)}
\end{eqnarray}

Now this is a product of all the numbers $f(e)^{\mu(d)}$ where $d$ and $e$ are natural numbers with $de | n$. Thus, we can exchange the places of $d$ and $e$ to obtain the following expression:
\begin{eqnarray}
\prod_{d|n} \hat{F}(d)^{\mu(n/d)} &=&  \prod_{e|n} \prod_{d| \frac{n}{e}} f(e)^{\mu(d)} \\
&=& \prod_{e|n} f(e)^{\sum_{d| \frac{n}{e}} \mu(d)} \\
&=& \prod_{e|n} f(e)^{h(n/e)} 
\end{eqnarray}

Where we have, by a lemma proven in class:
\begin{equation}
h(n/e) = \sum_{d | \frac{n}{e}} \mu(d) = \left\{ \begin{array}{c l} 
1 & \textrm{if } n/e = 1 \\
0 & \textrm{if } n/e > 1
\end{array} \right.
\end{equation}

This means that $f(e)^{h(n/e)} = f(n)$ when $e = n$ and $f(e)^{h(n/e)} = 0$ otherwise. This shows that $f(n) = \prod_{d|n} \hat{F} (d)^{\mu(n/d)}$, which is what we wanted to show. 
\end{sol}

\begin{prob}
Show that 
\begin{equation}
\prod_{\substack{a = 1\\(a,n) = 1}}^n a = n^{\phi(n)} \prod_{d|n} \legendre(d!, d^d)^{\mu(n/d)}
\end{equation}
\end{prob}

\begin{sol}
We know from the previous problem that if $\hat{F}(n) = \prod_{d|n} f(d)$, then $f(n) = \prod_{d|n} \hat{F}(d)^{\mu(n/d)}$. Thus, if we set $\hat{F}(d) = d!/d^d$, the proof of the problem is equivalent to proving that:
\begin{equation}
\frac{n!}{n^n} = \prod_{d|n} \frac{1}{d^{\phi(d)}} \prod_{\substack{a=1\\(a,n)=1}}^d a
\end{equation}

We shall use the prime factorization of $n = p_1^{e_1} p_2^{e_2} \ldots p_r^{e_r}$. Clearly, all factors $d|n$ have the form $p_1^{a_1} p_2^{a_2} \ldots p_r^{a_r}$ where $a_i \in \{0,1, \ldots, e_i \}$. We know that the sum $\sum_{d|n} \phi(d) = n$. Moreover, we know that $d  = p_1^{a_1} p_2^{a_2} \ldots p_r^{a_r}$. Thus, we can write $\prod_{d|n} d^{\phi(d)}$ as composed of their individual primes, exponentiated to some power. For $p_i$, we see that there are at least $\sum_{d|n} \phi(d) = n$ exponents, as well as some extra. Counting these, we find that $p_i$ has the exponents:
\begin{eqnarray}
p_i^{(\sum_{d|n} \phi(n))(e_i + 1)e_1 e_2 \ldots e_r } = p_i^{n (e_i+1) e_1 \ldots e_r}
\end{eqnarray}

Substituting this back into the equation, we find that
\begin{eqnarray}
\prod_{d|n} \frac{1}{d^{\phi(d)}} \prod_{\substack{a=1\\(a,n)=1}}^d a &=& \frac{1}{n^{n}} \frac{1}{p_1^{(e_1 + 1) \ldots e_r} \ldots p_r^{e_1 \ldots (e_r+1)}} \prod_{d|n}  \prod_{\substack{a=1\\(a,n)=1}}^d a \\
&=& \frac{n!}{n}
\end{eqnarray}

Which is what we wanted to show.
\end{sol}

\section{Problem 8}

Let $f,g$ be airthmetic functions.

\begin{prob}
Show that $Z(f * g, s) = Z(f,s) Z(g,s)$. 
\end{prob}

\begin{sol}
We shall expand out $Z(f*g,s)$ according to its definition:
\begin{eqnarray}
Z(f*g, s) &=& \sum_{n \geq 1} \frac{(f * g)(n)}{n^s} \\
&=& \sum_{n \geq 1} \sum_{d | n} \frac{1}{n^s} f(n/d) g(d) 
\end{eqnarray}

Now let us expand out $ Z(f,s) Z(g,s)$ according to its definition:
\begin{eqnarray}
 Z(f,s) Z(g,s) &=& \left( \sum_{n \geq 1} \frac{f(n)}{n^s} \right) \left( \sum_{n \geq 1} \frac{g(n)}{n^s} \right) \\
&=& \sum_{i \geq 1} \frac{f(1) g(i)}{1^s i^s} + \sum_{i \geq 1} \frac{f(2) g(i)}{2^s i^s} + \ldots \\
&=& \sum_{z \geq 1} \sum_{i \geq 1} \frac{f(z) g(i)}{z^s i^s} \\
&=& \sum_{z \geq 1} \sum_{i \geq 1} \frac{1}{(zi)^s} f(z) g(i)  
\end{eqnarray}

We use a change of variables and set $n = zi$ and $d = i$. Next, we notice that $i$ ranges over the set $\mathrm{N}$, while $d$ for $d | n$ ranges over the set $(zi)/d = (zi)/i = z$ where $z$ ranges over $\mathrm{N}$. Thus, $d | n$ ranges over $\mathrm{N}$ and we can rewrite the above expression as:
\begin{eqnarray}
 Z(f,s) Z(g,s) = \sum_{n \geq 1} \sum_{d | n} \frac{1}{n^s} f(n/d) g(d) 
\end{eqnarray} 

This is what we found that $Z(f*g,s)$ was equal to, so we have finished.
\end{sol}

\begin{prob}
Show that there is an arithmetic function $f^{-1}$ such that $f * f^{-1} = f^{-1} * f = 1$ if and only if $f(1) \neq 0$. 
\end{prob}

\begin{sol}
Let $f$ be an arithmetic function such that $f(1) \neq 0$. By definition, we want to show that $(f * f^{-1}) (1) = 1$ and $(f * f^{-1})(n) = 0$ for all $n > 1$. Thus, we want to find $f^{-1}$ such that $f(1)f^{-1}(1) = 1$ and $(f*f^{-1})(n) = \sum_{d|n} f^{-1} (d) f(n/d) = 0$ for all $n > 1$. We must show that there exists a unique solution $f^{-1}$. First, we know that $f^{-1}(1) = 1/f(1)$ will work when $n = 1$ because $f(1) \neq 0$. Moreover, notice that this is a unique value. Now, we shall proceed to show that all $f^{-1}(n)$ are unique by induction. Suppose that there exist unique values $f^{-1}(1), f^{-1}(2), \ldots f^{-1}(n-1)$. Since $f(1) \neq 0$, we can rewrite the condition that $\sum_{d|n} f^{-1} (d) f(n/d) = 0$ as the following:
\begin{eqnarray}
0 &=& f^{-1}(n)f(1) + \sum_{d|n, d<n} f^{-1} (d) f(n/d) \\
f^{-1}(n) &=& - \frac{1}{f(1)} \sum_{d|n, d<n} f^{-1} (d) f(n/d)
\end{eqnarray}

Since each $f^{-1}(x)$ for $x < n$ were unique, we can see that $f^{-1}(n)$ must also be unique. 

To show the converse, we note that if $f^{-1}(n)$ exists, then it must be of the form $f^{-1}(n) =  - \frac{1}{f(1)} \sum_{d|n, d<n} f^{-1} (d) f(n/d)$ since we have just shown that this inverse is unique. Since this exists and is well defined, we know that $f(1) \neq 0$. This completes the proof.
\end{sol}

\end{document}