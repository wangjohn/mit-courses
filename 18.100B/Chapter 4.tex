\documentclass[psamsfonts]{amsart}

%-------Packages---------
\usepackage{amssymb,amsfonts}
\usepackage[all,arc]{xy}
\usepackage{enumerate}
\usepackage{mathrsfs}
\usepackage[margin=1in]{geometry}


%--------Theorem Environments--------
%theoremstyle{plain} --- default
\newtheorem{thm}{Theorem}[section]
\newtheorem{cor}[thm]{Corollary}
\newtheorem{prop}[thm]{Proposition}
\newtheorem{lem}[thm]{Lemma}
\newtheorem{conj}[thm]{Conjecture}
\newtheorem{quest}[thm]{Question}

\theoremstyle{definition}
\newtheorem{defn}[thm]{Definition}
\newtheorem{defns}[thm]{Definitions}
\newtheorem{con}[thm]{Construction}
\newtheorem{exmp}[thm]{Example}
\newtheorem{exmps}[thm]{Examples}
\newtheorem{notn}[thm]{Notation}
\newtheorem{notns}[thm]{Notations}
\newtheorem{addm}[thm]{Addendum}
\newtheorem{exer}[thm]{Exercise}

\theoremstyle{remark}
\newtheorem{rem}[thm]{Remark}
\newtheorem{rems}[thm]{Remarks}
\newtheorem{warn}[thm]{Warning}
\newtheorem{sch}[thm]{Scholium}

\makeatletter
\let\c@equation\c@thm
\makeatother
\numberwithin{equation}{section}

\bibliographystyle{plain}

\voffset = -10pt
\headheight = 0pt
\topmargin = -20pt
\textheight = 690pt

%--------Meta Data: Fill in your info------
\title{Rudin Chapter 4\\
Solutions }

\author{John Wang}

\begin{document}

\maketitle


\section{Problem 4.1}

\begin{thm}
Suppose $f$ is a real function defined on $\mathbb{R}^1$ which satisfies $\lim_{h \to 0} [f(x+h) - f(x-h)] = 0$ for all $x \in \mathbb{R}^1$. This does not necessarily imply that $f$ is continuous.
\end{thm}

\begin{proof}
Consider the following function on $\mathbb{R}^1$:

\begin{equation}
f(x) = \left\{ \begin{array}{c c}
1 & \text{if } x = 0 \\
0 & \text{else}
\end{array} \right.
\end{equation}

We can see that $f(x)$ satisfies $\lim_{h \to 0} [f(x+h) - f(x-h)] = 0$ for all $x \neq 0$. This is because we can define $g_x(h) = f(x+h) - f(x-h)$ for each $x \in \mathbb{R}^1$. It is clear that for all $x \neq 0$ we have $\lim_{h \to 0} g_x(h) = 0$ because $f(x) = 0$ is a constant. Thus we are left to show that $\lim_{h \to 0} g_0(h) = 0$. Thus, we must obtain $\lim_{h \to 0} g_0(h+)$ and $\lim_{h \to 0} g_0(h-)$. If both of them are equal to zero, then we have shown that $f(x)$ satisfies the hypothesis given in the theorem.

Indeed, we can see that $\lim_{h \to 0} g_0(h+)=\lim_{h \to 0} g_0(h-)$. This is due, first, to the symmetry of $f(x)$ about $0$. Second, we know that $f(h) = f(-h)$ for all $h \neq 0$. Thus, $f(h) - f(-h) = 0$ and $f(-h) - f(h) = 0$ if $h \neq 0$. This shows that $\lim_{h \to 0} f(h) - f(-h) = \lim_{h \to 0} f(-h) - f(h) = 0$.

Finally, we can see that this function is not continuous. This is because $\lim_{x \to 0} f(x) = 0$ while $f(0) = 1$, which by a theorem in Rudin shows that $f(x)$ is not continuous at $x = 0$.  
\end{proof}

\section{Problem 4.2}

\begin{thm}
If $f$ is a continuous mapping of a metric space $X$ into a metric space $Y$, then $f(\bar{E}) \subset \bar{f(E)}$ for every set $E \subset X$. Moreover, $f(\bar{E})$ can be a proper subset of $\overline{f(E)}$. 
\end{thm}

\begin{proof}
Let $x_0 \in \bar{E}$. Then we must have $x_0 \in E$ or $x_0 \in E'$. If we assume that $x_0 \in E$, then we have $f(x_0) \in f(E)$. Moreover, since $x_0 \in \bar{E}$, we see that $f(\bar{E}) \subset \overline{f(E)}$. If $x_0 \in E'$, then there is a sequence $\{ p_n \}$ such that $\lim_{n \to \infty} f(p_n) = f(x_0)$ for every $\{ p_n \} \rightarrow x_0$ since $x_0$ is a limit point of $E$. Since every $ \{ f(p_n) \}$ is in $f(E)$, we know that $f(x_0) \in \overline{f(E)}$. 

To show that $f(\bar{E})$ can be a proper subset of $\overline{f(E)}$, take the function $f(x) = \frac{1}{x}$. Next, take the following interval $E = (1, \infty)$. We can see that $\bar{E} = [1,\infty)$ and that $f(E) = (0,1)$. Thus, $\overline{f(E)} = [0,1]$. However, we can see that $f(\bar{E}) = f([1,\infty]) = (0,1]$. It is easy to see that $f(\bar{E})$ is be a proper subset of $\overline{f(E)}$.
\end{proof}

\section{Problem 4.4}

\begin{thm}
Let $f$ and $g$ be continuous mappings of a metric space $X$ into a metric space $Y$ and let $E$ be a dense subset of $X$. Then $f(E)$ is dense in $f(X)$.
\end{thm}

\begin{proof}
We must show that each element of $f(X)$ is either in $f(E)$ or is a limit point of $f(E)$. Since we know $E$ to be dense in $X$, we know that each point in $X$ is either a point in $E$ or a limit point of $E$. If $p \in E$, then we know $f(p) \in f(X)$. Thus, we must show that for each limit point of $E$ not also in $E$, denoted by $p' \in E'$, we have a corresponding $f(p')$ that is a limit point of $f(E)$. 

To show this, we note that $f$ is continuous, which implies that for all $\epsilon > 0$, there exists a $\delta > 0$ such that $d( f(p), f(p')) < \epsilon$ if $d(p,p') < \delta$, where $p \in E$ and $p' \in E'$. In other words, $\forall \epsilon > 0$, there is a corresponding neighborhood $N_{\epsilon}( f(p'))$ with $f(p) \in f(E)$ and $f(p) \in N_{\epsilon}( f(p'))$. This shows that $f(p')$ is a limit point of $f(E)$. Since each $f(p')$ is a limit point of $f(E)$, and we know that $X$ consists entirely of $p$ and $p'$, we can see that $f(X)$ consits entirely of $f(p')$ and $f(p)$, which completes the proof.  
\end{proof}

\begin{thm}
If $g(p) = f(p)$ for all $p \in E$, then $g(p) = f(p)$ for all $p \in X$.
\end{thm}

\begin{proof}
Since $E$ is dense in $X$, we know that $ X = E \cup E^c$ and that each $p' \in E^c$ is a limit point of $E$. Since we assumed $g(p) = f(p)$ for all $p \in E$, we only need to show $g(p') = f(p')$ for all $p' \in E^c$. We know that there exists a sequence $\{ p_n \}$ in $E$ which converges to each $p' \in E^c$ because $p'$ is a limit point of $E$. Thus we know that $f(p_n) = g(p_n)$, because $p_n \in E$. Moreover by continuity of $f$ and $g$, we know that $\lim_{p_n \to p'} f(p_n) = f(p')$ and $\lim_{p_n \to p'} g(p_n) = g(p')$. Since $f(p_n) = g(p_n)$, we know that the limits are the same as well so that $f(p') = g(p')$. This completes the proof.
\end{proof}

\section{Problem 4.5}

\begin{thm}
If $f$ is a real continuous function defined on a closed set $E \subset \mathbb{R}^1$, then there exist continuous real functions $g$ on $\mathbb{R}^1$ such that $g(x) = f(x)$ for all $x \in E$. However, the result fails when the word ``closed'' is omitted from the hypothesis.
\end{thm}

\begin{proof}
Since a closed set on $\mathbb{R}^1$ must be a closed interval, we will let $E = [a,b]$. Next, consider the following function:

\begin{equation}
g(x) = \left\{ \begin{array}{c c}
f(a) & x \leq a \\
f(x) & a < x < b \\
f(b) & x \geq b 
\end{array} \right.
\end{equation}

We see that $g(x) = f(x)$ for all $x \in [a,b]$. Now, we will show that $g(x)$ is continuous on $\mathbb{R}^1$. First, we know that $g(x)$ is continuous on $(a,b)$ because $f(x)$ is a continuous function by assumption. We also know that $g(x)$ is continuous on $(-\infty,a) \cup (b, \infty)$ because $f(a)$ and $f(b)$ are constants, and constants are continuous functions. Thus, we must show that $g(x)$ is also continuous at points $x = a$ and $x = b$. To do this, we shall look at the left-hand and right-hand limits of $f(a)$. We see that $\lim_{x \to a^{-}} f(x) = f(a)$ because $f(x) = f(a)$ for $x \leq a$. We also know that $\lim_{x \to a^{+}} f(x) = f(a)$ because the continuity of $f(x)$ on $[a,b]$ implies that $f(a) = lim_{x \to a} f(x) = \lim_{x \to a^{+}} f(x)$. Thus, $g(x)$ is a continuous function on $\mathbb{R}^1$.

This result fails when the set $E$ is not closed. Consider for instance $E = (-\infty,0) \cup (0, \infty)$ and $f(x) = \frac{1}{x}$. Then we see that $\lim_{x \to 0^{-}} f(x) = -\infty$ and $\lim_{x \to 0^{+}} f(x) = + \infty$. This means that no value for $g(0)$ will be continuous if $g(x) = f(x)$ for all $x \in E$. This is because to be continuous, $g(0-) = - \infty$ must equal $g(0+) = + \infty$, which is impossible.
\end{proof}

\begin{thm}
Let $\vec{f} = (f_1, f_2, \ldots, f_k)$ be a real continuous vector valued function defined on a closed set $E \subset \mathbb{R}^k$. Then there exist continuous real functions $\vec{g}$ on $\mathbb{R}^k$ such that $\vec{g}(x) = \vec{f}(x)$ for all $\vec{x} \in E$. 
\end{thm}

\begin{proof}
If $E$ is a closed set in $\mathbb{R}^k$, then it must be composed of a k-cell of closed intervals $[a_1,b_1],[a_2,b_2],\ldots,[a_k,b_k]$. We can define $\vec{g}(x) = (g_1(x), g_2(x), \ldots, g_k(x))$ by defining the individual functions $g_n(x)$, $n = 1,2, \ldots, k$ as the following:

\begin{equation}
g_n(x) = \left\{ \begin{array}{c c}
f_n(a_n) & x \leq a_n \\
f_n(x) & a_n < x < b_n \\
f_n(b_n) & x \geq b_n 
\end{array} \right.
\end{equation}

We know that $f_1(x), f_2(x), \ldots, f_k(x)$ are all continuous on $[a_1,b_1],[a_2,b_2],\ldots,[a_k,b_k]$ respectively by the fact that $\vec{f}$ is continuous. Thus, using the argument from the above theorem, we can see that each $g_n(x)$ is continuous. Thus, $\vec{g}$ is continuous because each of its component functions is also continuous.
\end{proof}

\section{Problem 4.8}

\begin{thm}
If $f$ is a real uniformly continuous function on the bounded set $E \subset \mathbb{R}^1$, then $f$ is bounded in $E$.
\end{thm}

\begin{proof}
If $E$ is bounded, then we know that $\bar{E}$ is also bounded. This is because for each limit point $p \in E'$, there is a point $x \in E$ such that $d(x,p) < \epsilon$ for all $\epsilon > 0$. Since $E$ is bounded, there exists an $M>0$ such that $d(x,q) < M$ for all $x \in E$ and for some $q \in \mathbb{R}$. By the triangle inequality, $d(p,q) \leq d(x,p) + d(x,q)$ for some $p \in \bar{E}$. Thus we see that $d(p,q) \leq \epsilon + M$, which shows that $\bar{E}$ is bounded. 

Thus, we see that $\bar{E}$ is compact by Heine-Borel because it is both closed and bounded. Since $f$ is uniformly continuous on $E$, its extension to $\bar{E}$, let us call it $\bar{f}$, must also be continuous on $\bar{E}$. This is because for some limit point $p \in E'$, we have $d(p,x) < \rho$ for all $\rho > 0$ and $x \in E$ by the definition of limit point. Thus, we can pick a $\rho < \delta$ such that $d(p,x) < \rho < \delta$, implying that $d(f(p),f(x)) < \epsilon$ for all $\epsilon > 0$ by uniform continuity. 

Since $\bar{f}$ is a continuous function on a compact set $\bar{E}$, we see that its image $\bar{f}(\bar{E})$ is also a compact set in $\mathbb{R}^1$. In particular, we know that it is bounded. Moreover, since $f(E) \subset \bar{f}(\bar{E})$, we know that $f(E)$ is bounded. 
\end{proof}

\begin{thm}
If $E \subset \mathbb{R}^1$ is not bounded, then $f$ is not bounded in $E$. 
\end{thm}

\begin{proof}
Take the function $f(x) = x$ and the set $E = \mathbb{R}^1$. We can see that $f(x) = x$ is a real function that is uniformly continuous because for $\epsilon > 0$ there exists $\delta > 0$ such that $|f(x) - f(y)| < \epsilon$ if $|x - y| < \delta$ for all $x,y \in \mathbb{R}^1$. This is because for any $\epsilon > 0$, we can pick $\delta = \epsilon$ to satisfy the inequalities, since $f(x) = x$ and $f(y) = y$. 

Moreover, it is clear that $f$ is not bounded in $E$, because $E$ itself is not bounded and the range of $f$ is given by $E$.  
\end{proof}

\section{Problem 4.14}

\begin{thm}
Let $I = [0,1]$ be the closed unit interval. Suppose $f$ is a continuous mapping of $I$ onto $I$. Then $f(x) = x$ for at least one $x \in I$. 
\end{thm}

\begin{proof}
Define $g(x) = f(x) - x$. Then we can see that $g(x)$ is continuous on $[0,1]$ because $f(x)$ is continuous by assumption, $x$ is continuous as a polynomial, and $(h-j)(x)$ of two continuous functions $h$ and $j$ is also continuous. Notice that if we have $g(0) = 0$ or $g(1) = 0$, then the proof is completed.

Otherwise, we must have $g(0) = f(0) - 0 > 0$ and $g(1) = f(1) - 1 < 0$ because $f(x)$ is mapped onto $I = [0,1]$. Since $g(0) > 0$ and $g(1) < 0$, we can use the intermediate value theorem to show that there must be some $z$ in the interval $[0,1]$ such that $g(z) = 0$ because $g$ is continuous. Thus, we see that $f(z) - z = 0$ or $f(z) = z$ for some $z \in I$. This completes the proof.
\end{proof}

\section{Problem 4.15}

\begin{thm}
Call a mapping of $X$ into $Y$ open if $f(V)$ is an open set in $Y$ whenever $V$ is an open set in $X$. Then every continuous open mapping of $\mathbb{R}^1$ into $\mathbb{R}^1$ is monotonic.
\end{thm}

\begin{proof}
First we will prove two lemmas for a continuous open mapping $f: \mathbb{R}^1 \rightarrow \mathbb{R}^1$. 

\begin{lem}
If $a \neq b$, then $f(a) \neq f(b)$. 
\end{lem}

\begin{proof}
Let $f$ be a mapping in $[a,b]$. Then we know that $[a,b]$ is compact because it is real, closed, and bounded. Also, we know that for every continuous function on a compact space, $M = \sup_{x \in [a,b]} f(x)$ and $m = \inf_{x \in [a,b]} f(x)$ exist. There must be some values $x_1, x_2 \in [a,b]$ such that $f(x_1)$ is a maximum and $f(x_2)$ is a minimum. The case where $M = m$ implies that $f(x)$ is constant on $[a,b]$. Since a finite point is not open, then $f(x)$ does not satisfy the hypothesis of an open mapping for $M = m$. We must assume then that $M > m$.

Now, if $f(x_1) = m$ for some $x_1 \in (a,b)$, then $f$ is not open on $(a,b)$. This is because every neighborhood $N_r(f(x_1))$ contains values $p < m$ such that $p \notin f((a,b))$. Thus, for no value of $r$ is $N_r(f(x_1)) \subset f((a,b))$, which means that $f((a,b))$ is not open for this case.

The same applies for $f(x_2) = M$ for some $x_2 \in (a,b)$. Every neighborhood $N_r(f(x_2))$ contains values $p > M$ such that $p \notin f((a,b))$. Thus, for no value of $r$ is $N_r(f(x_2)) \subset f((a,b))$, which means that $f((a,b))$ is not open for this case. 

Thus, there are only two cases left. First, we could have $f(a) = m$ and $f(b) = M$, which would imply that $f(a) < f(b)$ and thus that $f(a) \neq f(b)$. Second, we could have $f(a) = M$ and $f(b) = m$, which would imply that $f(a) > f(b)$ and thus that $f(a) \neq f(b)$. This completes the proof of the lemma.
\end{proof}

Now, we will proceed to the prove the second lemma.

\begin{lem}
If $a < b < c$ and $f(a) < f(b)$, then $f(b) < f(c)$. 
\end{lem}

\begin{proof}
First, notice that if $f(c) = f(a)$ or $f(c) = f(b)$, this would contradict the previous lemma. Thus, there are three cases left that could possibly occur. The first case is where $f(a) < f(b)$ and $f(a) < f(c) < f(b)$. Using the intermediate value theorem, there must exist some $x \in (a,b)$ such that $f(x) = f(c)$. Since $c \notin (a,b)$, this would mean that $x \neq c$, but that $f(x) = f(c)$, which is a contradiction of our previous lemma.

The second case is if $f(c) < f(a)$. In this case, we have $f(c) < f(a) < f(b)$. Using the intermediate value theorem, there must exist some $x \in (b,c)$ such that $f(x) = f(a)$. Since $a \notin (b,c)$, we must have $x \neq a$ and $f(x) = f(a)$, which is a contradiction of our previous lemma.

Thus, the only possibility left is $f(a) < f(b) < f(c)$. Since the real field $R$ is ordered, we know that this last possibility must always hold.
\end{proof}

Now, we can proceed to prove the theorem. Assume by contradiction that an open mapping $f$ from $\mathbb{R}^1$ to $\mathbb{R}^1$ is not monotonically increasing or decreasing. Then for points $x_1 < x_2 < x_3$ in $\mathbb{R}^1$, we must have either $f(x_1) < f(x_2)$ and $f(x_3) < f(x_2)$ or $f(x_1) > f(x_2)$ and $f(x_3) > f(x_2)$. The first case is clearly a contradiction of our second lemma. For the second case, let us use $g(x) = -f(x)$ to obtain the following relations: $g(x_1) < g(x_2)$ and $g(x_3) < g(x_2)$. We know that $g(x)$ is still continuous because we have simply multiplied a continuous function by a constant. We also know that $g(x)$ is an open mapping. This is because $f(V)$ is open for every open set $V \in \mathbb{R}^1$, implying that $-f(V)$ is also open. Thus, we have found a continuous open mapping $g(x)$ which is a contradiction of our second lemma. Therefore, our assumption must be incorrect, and each open mapping from $\mathbb{R}^1$ to $\mathbb{R}^1$ is monotonic. 
\end{proof}

\section{Problem 4.16}

\begin{thm}
Let $[x]$ denote the largest integer contained in $x$, that is $[x]$ is the integer suh that $x-1 < [x] \leq x$; and let $(x) = x - [x]$ denote the fractional part of $x$. Then $f(x) = [x]$ and $g(x) = (x)$ are discontinuous for all $x \in \mathbb{Z}$ and continuous everywhere else.
\end{thm}

\begin{proof}
To begin, we shall show that $f(x)$ and $g(x)$ are continuous at all points $x \notin \mathbb{Z}$. We know that there exists an integer $n$ such that $x \in (n, n+1)$ if $x$ is not an integer. Thus, we see that $f(x) = n$ and $g(x) = x - n$. Since $n$ is a constant, $f(x)$ is continuous. Moreover, since $x$ is a polynomial, which is continuous, and $(h + j)(x)$ is continuous if $h$ and $j$ are continuous, we can see that $g(x) = x - n$ is also continuous. This shows that for non-integer values of $x$, $f(x)$ and $g(x)$ are continuous.

To show that $f(x)$ and $g(x)$ are discontinuous for $x \in \mathbb{Z}$, we will compute the right-hand and left-hand limits of both functions. We see that as x approaches $n \in \mathbb{Z}$, we have the following limits:

\begin{eqnarray}
\lim_{x \to n^{+}} [x] &=& n \\
\lim_{x \to n^{-}} [x] &=& n - 1 \\
\lim_{x \to n^{+}} (x) &=& 0 \\
\lim_{x \to n^{-}} (x) &=& 1
\end{eqnarray}

We see that $f(n+) \neq f(n-)$ and $g(n+) \neq g(n-)$. This implies that $[x]$ and $(x)$ are not continuous when $x \in \mathbb{Z}$. 
\end{proof}

\section{Problem 4.22}

\begin{thm}
Let $A$ and $B$ be disjoint nonempty closed sets in a metric space $X$, and define $f(p) = \frac{\rho_A(p)}{\rho_A(p) + \rho_B(p)}$, where $p \in X$ and $\rho_E(p) = \inf_{z \in E} d(p,z)$. Then $f$ is a continuous function on $X$ whose range lies in $[0,1]$. 
\end{thm}

\begin{proof}
We will show that $\rho_E(p)$ is continuous and that $\rho_E(p) = 0$ if and only if $p \in \bar{E}$. This will allow us to prove that $f$ is a continuous function by the rules for the addition of continuous functions and because $A$ and $B$ are disjoint, nonempty, and closed. Thus, we will need two lemmas.

\begin{lem}
If $\rho_E$ is defined as above, then $\rho_E(p) = 0$ if and only if $p \in \bar{E}$.
\end{lem}

\begin{proof}
Suppose $\rho_E(p) = 0$, then $0 = \inf_{z \in E} d(p,z)$. Suppose by contradiction that $p \notin \bar{E}$ so that $p$ is neither a limit point of $E$ nor a point in $E$. Then for some $r > 0$, there does not exist a neighborhood $N_r(p)$ such that $z \in E$ and $z \in N_r(p)$. This would imply that $d(p,z) > r$, and that $r = \inf_{z \in E} d(p,z)$. Since $r > 0$ implies $r \neq 0$, we obtain a contradiction with the hypothesis that $\inf_{z \in E} d(p,z) = 0$. 

For the converse, assume that $p \in \bar{E}$. Then we have either $p \in E$ or $p \in E'$. The first case is trivial because for some $z \in E$, we must have $d(p,z) = 0$. If $p \in E'$, then for every $\epsilon >0$, we have $d(p,z) < \epsilon$. Since $\epsilon$ is arbitrary, we have $\inf_{z \in E} d(p,z) = 0$. 
\end{proof}

\begin{lem}
If $\rho_E$ is defined as above, then $\rho_E$ is uniformly continuous.
\end{lem}

\begin{proof}
Since we have defined $\rho_E(p) = \inf_{z \in E} d(p,z)$, we can use the triangle inequality to show that $\inf_{z \in E} d(p,z) \leq \inf_{z \in E} d(p,y) + d(y,z)$ where $y \in X$. Moreover, since we only obtain the infimum of $d(p,y) + d(y,z)$ when we have the infimum of both $d(p,y)$ and $d(y,z)$, then we can see that $\rho_E(p) \leq d(p,y) + \rho_E(y)$. We can rearrange this to show that $|\rho_E(p) - \rho_E(y) | \leq d(p,y)$. Thus, for any $\epsilon > 0$, we can always find $\delta = \epsilon$ such that $|\rho_E(p) - \rho_E(y)| \leq d(x,y) < \delta = \epsilon$. This shows that $\rho_E$ is uniformly continuous.
\end{proof}

Thus, we have shown that $\rho_E$ is uniformly continuous, and thus continuous by the second lemma. Since $\rho_A$ and $\rho_B$ are both continuous, we know that $\rho_A + \rho_B$ is continuous. Moreover, we know that $f$ is continuous at all points except possibly where $\rho_A(p) + \rho_B(p) = 0$, by the rule for the division of continuous functions. However, we know that $A$ and $B$ are disjoint, nonempty closed sets so that $\bar{A} = A$ and $\bar{B} = B$. This means that if $x \in A$, then $x \in \bar{A}$. Since the two sets are disjoint, then $x \notin B$ implies $x \notin \bar{B}$. By the first lemma, if $\rho_A(p) = 0$, then $\rho_B(p) \neq 0$ and vice versa. Thus, $\rho_A(p) + \rho_B(p) \neq 0$ for any value of $p \in X$. This means that $f$ is continuous everywhere. 

Next, we will show that $f$ has a range $[0,1]$. First, we know that $\rho_A(p) \leq \rho_A(p) + \rho_B(p)$, since $\rho_B(p) \geq 0$ by the fact that a metric must be non-negative. This gives an upper bound of $f(p) \leq 1$. Next, since $\rho_A(p) \geq 0$ by the same fact that metrics must be non-negative, we know that $\rho_A(p) + \rho_B(p) \geq 0$ and so that $\frac{\rho_A(p)}{\rho_A(p) + \rho_B(p)} \geq 0$. This gives a lower bound $f(p) \geq 0$, and since $f$ is continuous, it has a range of $[0,1]$. 
\end{proof}

\begin{thm}
Let $f$ be defined as above, then $f(p) = 0$ precisely on $A$ and $f(p) = 1$ precisely on $B$. 
\end{thm}

\begin{proof}
We have shown above in the first lemma that $\rho_E(p) = 0$ if and only if $p \in \bar{E}$. Thus, the only time that $\rho_A(p) = 0$ is when $p \in \bar{A}$ and since $A$ is closed, this is equivalent to $p \in A$. Since $f(p) = \frac{\rho_A(p)}{\rho_A(p) + \rho_B(p)}$, the only time that $f(p) = 0$ is when $\rho_A(p) = 0$, which is when $p \in A$. 

Moreover, we know that the only time $f(p) = 1$ is when $\rho_B(p) = 0$ and $\rho_A(p) \neq 0$ because then $f(p) = \frac{\rho_A(p)}{\rho_A(p)} = 1$. This occurs when $p \in \bar{B}$, and since $\bar{A} \cap \bar{B} = \emptyset$, we can be sure that $p \notin \bar{A}$. Thus, $f(p) = 1$ when $p \in B$, since $B$ is closed already.  
\end{proof}

\begin{thm}
Set $V = f^{-1}([0,\frac{1}{2}))$ and $W = f^{-1}((\frac{1}{2},1])$, then $V$ and $W$ are open and disjoint, and $A \subset V$, $B \subset W$. 
\end{thm}

\begin{proof}
First, we can see that $[0, \frac{1}{2})$ and $(\frac{1}{2}, 1]$ are open sets in $f(X)$ because the range of $f(X)$ is $[0,1]$. Since $f$ is continuous, we know that the sets $f^{-1}([0, \frac{1}{2}))$ and $f^{-1}((\frac{1}{2},1])$ are also open. Moreover, these two sets are disjoint because each $f(p)$ maps to a single value, so that $f^{-1}(f(p))$ belongs to only one of $V$ or $W$, which ensures that $V \cap W = \emptyset$. We know that $f(p) = 0$ when $p \in A$ and that $A = f^{-1}(0) \subset  f^{-1}([0, \frac{1}{2})) = V$ and since $f(p) = 1$ when $p \in B$, we see that $B = f^{-1}(1) \subset f^{-1}((\frac{1}{2},1]) = W$. Thus, we have $A \subset V$ and $B \subset W$. 
\end{proof}

\section{Problem 4.23}

\begin{thm}
A real valued function $f$ defined in $(a,b)$ is said to be convex if $f(\lambda x + ( 1 - \lambda) y) \leq \lambda f(x) + (1 - \lambda) f(y)$ whenever $a<x <b, a<y<b, 0 < \lambda < 1$. We will prove the last proposition first, namely that if $f$ is convex in $(a,b)$ and if $a<s<t<u<b$, then 
\begin{equation}
\frac{f(t) - f(s)}{t-s} \leq \frac{f(u) - f(s)}{u-s} \leq \frac{f(u) - f(t)}{u -t}
\end{equation}
\end{thm}

\begin{proof}
We can let $t = \lambda s + (1 - \lambda) u$ where $\lambda = \frac{u-t}{u-s}$. Using this definition, it is easy to see that $0< \lambda < 1$. Thus, we have the following:
\begin{eqnarray}
f(t) &=& f(\lambda s + (1-\lambda)u) \leq \lambda f(s) + (1 - \lambda) f(u) \\
0 &\leq& - f(t) + \frac{u-t}{u-s} f(s) + \left(1 - \frac{u-t}{u-s} \right) f(u) \\
0 &\leq& -(u - s) f(t) + (u - t) f(s) + (t - s) f(u). \label{eq1}
\end{eqnarray}

We can see that if we rearrange the terms in the above inequality and add $ s f(s) - s f(s) $ to the right side, we obtain:

\begin{eqnarray}
0 &\leq& (u - s) (f(s) - f(t)) + s f(s) - t f(s) + (t - s) f(u) \\
0 &\leq& -(u - s) (f(t) - f(s)) + (t - s)( f(u) - f(s))
\end{eqnarray}
\begin{equation}
\frac{f(t) - f(s)}{t-s} \leq \frac{f(u) - f(s)}{u-s} 
\end{equation}

Which completes the first inequality that we wanted to show. To show the second inequality, we return to inequality (10.5) and add $ u f(u) - u f(u)$ to the right hand side.

\begin{eqnarray}
0 &\leq& (u - s)(f(u) - f(t)) - u f(u) + (u - t) f(s) + t f(u) \\
0 &\leq& (u - s)(f(u) - f(t)) - (u - t)(f(u) - f(s))
\end{eqnarray}
\begin{equation}
\frac{f(u) - f(s)}{u - s} \leq \frac{f(u) - f(t)}{u - t}
\end{equation}

Putting these two inequalities together, we obtain what we wanted to prove.
\end{proof}

\begin{thm}
If $f$ is a convex function, then it is continuous.
\end{thm}

\begin{proof}
Let $f$ be defined on the interval $(a,b)$ and let $x \in (a,b)$ be an element of $f$. Define $\delta >0$ such that $(x - \delta, x + \delta) \subset (a,b)$. Now consider a point $z \in (x - \delta, x)$. Then we have $a < z < x < x + \delta < b$. Thus, using the above inequalities, we find:

\begin{equation}
\frac{f(x) - f(z)}{x - z} \leq \frac{f(x + \delta) - f(z)}{(x + \delta) - z} \leq \frac{f(x + \delta) - f(x)}{(x + \delta) - x}
\end{equation}

Since we also have $a < x - \delta < z < x < b$, we can use the previously derived inequalities to obtain:

\begin{equation}
\frac{f(z) - f(x - \delta)}{z - (x - \delta)} \leq \frac{f(x) - f(x - \delta)}{x - (x - \delta)} \leq \frac{f(x) - f(z)}{x - z}
\end{equation}

Combining the above inequalities, we can derive:

\begin{equation}
\frac{f(x) - f(x - \delta)}{\delta} \leq \frac{f(x) - f(z)}{x - z} \leq \frac{f(x + \delta ) - f(x)}{\delta}
\end{equation}

These inequalities hold similarly for $z \in (x, x + \delta)$ by symmetric arguments. Thus, since we have fixed $x$ and $\delta$, we see that $c_1 (x - z) \leq f(x) - f(z) \leq c_2 (x-z)$, for constants $c_1$ and $c_2$. This further implies, for all $z \in (x - \delta, x + \delta)$:

\begin{equation}
| f(x) - f(z) | \leq C (x - z)
\end{equation}

Where $C$ is a constant. We have shown that if $| x - z| < B$, then there exists some number $A$ where $| f(x) - f(z) | < A$, which shows that a convex function is continuous.
\end{proof}

\begin{thm}
Every increasing convex function of a convex function is convex.
\end{thm}

\begin{proof}
Define $f(x)$ be a convex function defined on $(a,b)$ and $g(x)$ be an increasing convex function on $(c,d)$ where $(a,b) \subset (c,d)$. We must show that $h(x) = g(f(x))$ is convex. Since $f(x)$ is convex, we know that $f(\lambda x + (1 - \lambda) y) \leq \lambda f(x) + ( 1 - \lambda) f(y)$ for $x,y \in (a,b)$ and $0 < \lambda < 1$. Since $g(x)$ is increasing, we know that $g(f(\lambda x + (1 - \lambda) y)) \leq g(\lambda f(x) + ( 1 - \lambda) f(y))$. Thus, since $g(x)$ is also convex, we have:

\begin{eqnarray}
h(\lambda x + (1 - \lambda) y) &=& g(f(\lambda x + (1 - \lambda) y)) \\
& \leq & g(\lambda f(x) + ( 1 - \lambda) f(y)) \\
& \leq & \lambda g(f(x)) + (1 - \lambda) g(f(y)) \\
& \leq & \lambda h(x) + (1 - \lambda) h(y) 
\end{eqnarray}

This proves the convexity of $h(x)$. 
\end{proof}

\section{Problem 4.24}

\begin{thm}
Assume that $f$ is a continuous real function defined in $(a,b)$ such that $f( \frac{x +y}{2}) \leq \frac{ f(x) + f(y)}{2}$ for all $x,y \in (a,b)$, then $f$ is convex. 
\end{thm}

\begin{proof}
First, we will use mathematical induction to show the convexity of $f$ for every $\lambda$ of the form $\frac{m}{2^n}$ where $0 \leq m \leq 2^n$ and $m,n \in \mathbb{Z}_+$. Then, we will show that every $\lambda \in \mathbb{R}$ has a sequence of numbers of the form $\frac{m}{2^n}$ converging to it. 

So, let $\lambda = \frac{m}{2^n}$, $0 \leq m \leq 2^n$, and $m,n \in \mathbb{Z}_{+}$. Now start with the base case of $n=1$. The possible values of $m$ are given by $m = 0,1,2$. It is trivial to show convexity for $m = 0$ and $m = 2$, since $\lambda = 0$ or $\lambda = 1$ respectively. For $m = 1$, we have:

\begin{eqnarray}
f(\lambda x + (1 - \lambda) y) = f \left(\frac{x + y}{2} \right) \leq \frac{f(x) + f(y)}{2} = \lambda f(x) + (1 - \lambda) f(y)
\end{eqnarray}

Thus, we have established convexity of $f$ for $n = 1$. Now assume we have shown convexity of $f$ for $n = k$. We shall proceed to show that $f$ is convex for $n = k+1$. We can rewrite $\lambda$ as the following:

\begin{equation}
\lambda = \frac{m}{2^{k+1}} = \frac{1}{2} \left( \frac{m-1}{2^k} + \frac{1}{2^k} \right)
\end{equation}

Letting $\lambda_1 = \frac{m-1}{2^k}$ and $\lambda_2 = \frac{1}{2^k}$, we see that $\lambda = \frac{1}{2}(\lambda_1 + \lambda_2)$. We have assumed $f$ is convex for $\lambda_1 = \frac{m-1}{2^k}$ and $\lambda_2 = \frac{1}{2^k}$ because we have assumed convexity for all $n = k$. Therefore, we know that $f(\lambda_1 x + (1 - \lambda_1) y ) \leq \lambda_1 f(x) + (1 - \lambda_1) f(y)$ and $f(\lambda_2 x + (1- \lambda_2)y ) \leq \lambda_2 f(x) + (1 - \lambda_2) f(y)$. Thus, we have:

\begin{eqnarray}
f(\lambda x + (1 - \lambda) y) &=& f \left(\frac{\lambda_1 x + (1 - \lambda_1) y + \lambda_2 x + (1 - \lambda_2) y}{2} \right) \\
&\leq& \frac{1}{2} f(\lambda_1 x + (1 - \lambda_1) y) + \frac{1}{2} f( \lambda_2 x + (1 - \lambda_2) y) \\
&\leq& \frac{1}{2} \lambda_1 f(x) + \frac{1}{2} ( 1- \lambda_1) f(y) + \frac{1}{2} \lambda_2 f(x) + \frac{1}{2} (1 - \lambda_2) f(y) \\
&=& \frac{1}{2} ( \lambda_1 + \lambda_2) f(x) + (1 + \frac{1}{2} (\lambda_1 + \lambda_2)) f(y) \\
&=& \lambda f(x) + (1 - \lambda) f(y)
\end{eqnarray}

Thus, we have shown by induction that if $\lambda = \frac{m}{2^n}$ for $m,n \in \mathbb{Z}_+$ and $0 \leq m \leq 2^n$, then $f$ is convex. Now, we must show that $f$ is convex for any $\lambda \in \mathbb{R}$ with $0 < \lambda < 1$. We will show that every $\lambda \in \mathbb{R}$ in $0 < \lambda < 1$ has a sequence of $\{ \lambda_k \}$ that converges to $\lambda$. This is because each real number has a binary expansion (see the lemma at the bottom). This implies that there exists a sequence $\{ \lambda_k \}$ of the form $\frac{m}{2^n}$ such that $\{ \lambda_k \} \rightarrow \lambda$ as $k \rightarrow \infty$. This implies the that $f(\lambda x + (1 - \lambda) y) = \lim_{k \to \infty} f(\lambda_k x + (1 - \lambda_k) y) \leq \lim_{k \to \infty} \lambda_k f(x) + (1 - \lambda_k ) f(y) = \lambda f(x) + (1 - \lambda) f(x)$ for $\lambda \in (0,1)$. Thus, we have shown $f$ is convex for all $x,y \in (a,b)$ and $\lambda \in (0,1)$.
\end{proof}

\begin{lem}
Each real number $x \in (0,1)$ has a binary expansion.
\end{lem}

\begin{proof}
Since we have $x \in (0,1)$, we will construct $x_n \leq x < x_n + 2^{-n}$ and $a_n \in \{0,1\}$ such that $x_{n+1} = x_n + a_{n+1} 2^{-(n+1)}$. Thus we will have 

\begin{equation}
x_n = \sum_{i = 1}^n a_i 2^{-i}
\end{equation}

First, let $a_0 = x_0 = [x]$. Then since $x \in (0,1)$ and $a_0 \in \{0,1\}$, we have $x_0 \leq x < x_0 + \frac{1}{2}$. Now we use mathematical induction to construct the binary expansion. Assume we already have constructed $x_n$ so that $x_n \leq x < x_n + 2^{-n}$. Then we want to construct $x_{n+1}$. We know the interval $[x_n, x_n + 2^{-n})$ has a length $2^{-n}$ between its two endpoints. Let $x_{n+1} \in [x_n, x_n + 2^{-n})$ be the largest number of the form $x_n + a_{n+1} 2^{-(n+1)}$ which does not exceed $x$ and $a_{n+1} \in \{0,1\}$. Therefore, we know that $x_{n+1} = x_n + a_{n+1} 2^{-(n+1)}$ and that $x_{n+1} \leq x < x_n + (a_{n+1} + 1) 2^{-(n+1)} = x_{n+1} + 2^{-(n+1)}$.

Therefore, we have shown that $x \in [x_{n+1}, x_{n+1} + 2^{-(n+1)})$. Therefore, by induction, we have constructed $x_n$. To show that $\{x_n \}$ converges to $x$, we see that $\{x_n\}$ is monotonically increasing by construction and bounded from above by $x$. Therefore, there is a limit $p \in \mathbb{R}$. Also, we know that:

\begin{equation}
\lim_{n \to \infty} x_n \leq x \leq \lim_{n \to \infty} x + 2^{-n}
\end{equation}

And since $\lim_{n \to \infty} x_n = p$ and $\lim_{n \to \infty} x + 2^{-n} = p$, we have $p \leq x \leq p$ so that by the squeeze theorem, we must have $ p = x$. This shows that each real number $x \in (0,1)$ has a binary expansion.
\end{proof}

\end{document}