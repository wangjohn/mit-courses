\documentclass[psamsfonts]{amsart}

%-------Packages---------
\usepackage{amssymb,amsfonts}
\usepackage[all,arc]{xy}
\usepackage{enumerate}
\usepackage{mathrsfs}
\usepackage[margin=1in]{geometry}
\usepackage{thmtools}
\usepackage{verbatim}
\usepackage{multirow}
\usepackage{tikz}
\usepackage{hyperref}
\usetikzlibrary{shapes,arrows}


%--------Theorem Environments--------
%theoremstyle{plain} --- default
\newtheorem{prob}{Problem}[section]
\newtheorem{thm}{Theorem}[section]
\newtheorem{cor}[thm]{Corollary}
\newtheorem{prop}[thm]{Proposition}
\newtheorem{lem}[thm]{Lemma}
\newtheorem{conj}[thm]{Conjecture}
\newtheorem{quest}[thm]{Question}

\newenvironment{sol}{{\bfseries Solution}}{\qedsymbol}


\theoremstyle{definition}
\newtheorem{defn}[thm]{Definition}
\newtheorem{defns}[thm]{Definitions}
\newtheorem{con}[thm]{Construction}
\newtheorem{exmp}[thm]{Example}
\newtheorem{exmps}[thm]{Examples}
\newtheorem{notn}[thm]{Notation}
\newtheorem{notns}[thm]{Notations}
\newtheorem{addm}[thm]{Addendum}
\newtheorem{exer}[thm]{Exercise}

\theoremstyle{remark}
\newtheorem{rem}[thm]{Remark}
\newtheorem{rems}[thm]{Remarks}
\newtheorem{warn}[thm]{Warning}
\newtheorem{sch}[thm]{Scholium}

\makeatletter
\let\c@equation\c@thm
\makeatother
\numberwithin{equation}{section}

\bibliographystyle{plain}

\voffset = -10pt
\headheight = 0pt
\topmargin = -20pt
\textheight = 690pt

%--------Meta Data: Fill in your info------
\title{6.046 \\
Problem Set 4}

\author{John Wang}

\begin{document}

\maketitle

Collaborators: 

\section{Problem 5-1: High Probability Bounds on Randomized Select}

\subsection{Problem A}

\begin{prob}
Let $b$ be a real number such that $1/2 < b < 1$ and let $A_i$ be the array in the $i$th recursion. Define a bad pivot choice in the $i$th recursion as the one that results in $|A_{i+1}| > b |A_i|$. Give a lower bound on the probability of having $k$ bad pivot choices in a row.
\end{prob}

\begin{sol}
First, let us imagine the $A_i$ in sorted order. In order for a pivot to be bad so that $|A_{i+1}| > b |A_i|$, then we need either the smaller or larger secondary array to be larger than $b|A_{i}|$. This occurs if we pick a pivot with rank smaller than $bn$ or with rank greater than $n - (bn)$. If we select a pivot uniformly at random, then there is a $(1-b)$ chance of selecting a pivot in one of these regions. Since there are two regions, then we have a probability of $2(1-b)$ of selecting a bad pivot on the $i$th trial. Since each one of these trials is independent, the probability of selecting $k$ bad pivots in a row is then at least $(2(1-b))^k$. 
\end{sol}

\subsection{Problem B}

\begin{prob}
If our initial array size is $n$, then after one bad pivot choice, the next array is of size at least $bn$. Recall that the running time at each recursive call requires time equal to at least the size of the array. Give a precise lower bound on the total running time of $k$ recursive calls to \texttt{Randomized-Select} if in every recursive call we choose a bad pivot.
\end{prob}

\begin{sol}
Let $T$ be the running time of $k$ recursive calls. We know that $T$ is bounded by the following:
\begin{eqnarray}
T > \sum_{i=0}^{k-1} n b^i 
\end{eqnarray}

This follows because at the $i$th step in the recursion, we have an array of size at least $nb^i$. Therefore, since the running time at each recursive call requires time equal at least to the size of the array, we must have $T$ be greater than the sum of all the $k$ recursive calls. We can evaluate this sum to find:
\begin{eqnarray}
T &>& n \sum_{i=0}^{k-1} b^i \\
T &>& n \left( \frac{1 - b^{k-1}}{1-b} \right) 
\end{eqnarray}

Where we have used the formula for a partial geometric series. This gives us a lower bound on the running time of $k$ recursive calls if we choose a bad pivot each time. 
\end{sol}

\subsection{Problem C}

\begin{prob}
Use a proof by contradiction to disprove the following statement: Let $T(n)$ be the running time of \texttt{Randomized-Select} on an input of size $n$. Then there exist integers $n_0, c \geq 1$ such that for all $n \geq n_0$, $P(T(n) > cn) \leq 1/n$. 
\end{prob}

\begin{sol}
Suppose this statement were true. Then select $n_0, c$ such that they satisfy the conditions of the statement. Now let us examine the event $E$ defined as the event where \texttt{Randomized-Select} chooses $n$ bad pivots consecutively. Now let us choose $b = 1 - \frac{1}{2c}$ for a bad pivot. We know that the running time is bounded by the following:
\begin{eqnarray}
T(E) &>& n \left( \frac{1 - b^{n-1}}{1 - b} \right) \\
&=& n \left( \frac{1 - (1 - \frac{1}{2c})^{n-1}}{\frac{1}{2c}} \right) \\
&=& 2nc \left(1 - \left(\frac{2c - 1}{2c}\right)^{n-1} \right)
\end{eqnarray} 

Now let us assume that $T(E) > cn$ for some $n$. Then we must have the following hold:
\begin{eqnarray}
2nc \left(1 - \left(\frac{2c - 1}{2c}\right)^{n-1} \right) &>& cn \\
1 - \left(\frac{2c - 1}{2c}\right)^{n-1}&>& \frac{1}{2} \\
\left(\frac{2c - 1}{2c}\right)^{n-1} &<& \frac{1}{2}
\end{eqnarray}

Since we know that $\frac{2c-1}{2c} < 1$ (as $c > 1$ by assumption), we can take the log base $\frac{2c - 1}{2c}$ of both sides but must switch the inequality. This yields:
\begin{eqnarray}
n - 1 &>& \log_{\frac{2c - 1}{2c}} \left(\frac{1}{2} \right) \\
n &>& 1 - \log_{\frac{2c - 1}{2c}}(2)
\end{eqnarray} 

However, we know that the following holds true:
\begin{eqnarray}
0 < \frac{1}{c} < \frac{2c-1}{2c} < 1
\end{eqnarray}

This means that $\log_{\frac{2c-1}{2c}}(x) < \log_{\frac{1}{c}}(x)  <  0$ for $x > 1$. Therefore, we can give another bound for $n$ with:
\begin{eqnarray}
n &>& 1 - \log_{\frac{1}{c}}(2)
\end{eqnarray}

This means that if $T(E) > cn$, we must have $n > 1 - \log_{\frac{1}{c}}(2)$. Notice that since $\frac{1}{c} < 1$, we must have $ \log_{\frac{1}{c}}(2) < 0$, which means that $n > 1$. Now, if this is the case, let us examine the probability that we obtain $n$ bad recursive calls. This can be given by our solution to problem A:
\begin{eqnarray}
P[E] &>& (2(1-b))^{n} \\
&>& \left(2 \left(1- \left(1-\frac{1}{2c} \right) \right) \right)^{1 - \log_{{\frac{1}{c}}}(2)} \\
&=& \left(\frac{1}{c} \right)^{1 - \log_{\frac{1}{c}}(2)} \\
&=& \left(\frac{1}{c} \right) \frac{1}{2} \\
&=& \frac{1}{2c}
\end{eqnarray}

Therefore, we know that $P[E] > 1/(2c)$. However, notice that this is invariant with $n$. Therefore, if we select $n$ large enough such that $n > 1 - \log_{\frac{1}{c}}(2)$ and such that $1/n > 1/(2c)$, we we see that $P[E] \not \leq 1/n$. Therefore, we can select $n > 2c + (1 - \log_{\frac{1}{c}}(2)) + n_0$ such that we are sure that $n > n_0$, $\frac{1}{n} > \frac{1}{2c}$, and $T[E] > cn$. Since we can choose $n$ arbitrarily large to satisfy this, we have found an $n$ such that $P(T(n) > cn) \not \leq \frac{1}{n}$, which is a contradiction of our statement. This completes the proof.
\end{sol}

\section{Problem 5-2: Random Vectors and Matrices}

\subsection{Problem A}

\begin{prob}
You are given a non-zero vector $\vec{u} \in \mathrm{Z}_p^n$ and some number $c \in \mathrm{Z}_p$. Prove that if another vector $\vec{v} \in \mathrm{Z}_p^n$ has each element chosen independently and uniformly at random from $\mathrm{Z}_p$, then the probability that $\vec{v} \cdot \vec{u} = c $ is $1/p$. 
\end{prob}

\begin{sol}
First, we shall show it works for the base case of $n = 1$, then use induction on $n$. Therefore, we first will show that if $\vec{u} \in \mathrm{Z}_p^1$ is an integer $u_1 \in \mathrm{Z}_p$, and we pick another number $v_1 \in \mathrm{Z}_p$ uniformly at random, the the probability that $u_1 \cdot v_1 \equiv c \pmod{p}$ is $1/p$. To show this, we prove the following lemma:

\begin{lem}
For any number $u \in \mathrm{Z}_p$, there exists a unique integer $\bar{u}_c \in \mathrm{Z}_p$ such that $u \cdot \bar{u}_c \equiv c \pmod{p}$. 
\end{lem}

\begin{proof}
First we will show existence. Let $g$ be a primitive root modulo $p$, which exists because $p$ is a prime. Next, we know that $1,g,g^2, \ldots, g^{p-2}$ is a reordering of $1,2,3, \ldots, p-1$ modulo $p$ by the definition of primitive root. We can express $u = g^k$ for some $k \in \{1,2,\ldots, p-2\}$ and $c = g^m$ for some $m \in \{1,2,\ldots, p-2\}$. Now if $m > k$, we can simply let $\bar{u}_c = g^{m - k}$ so that $u \cdot \bar{u}_c = g^{k} g^{m - k} = g^{m} \equiv c \pmod{m}$. If $m = k$, we let $\bar{u}_c = 1$. If $m < k$, then we choose $\bar{u}_c = g^{m + (p-1) - k}$ so that $u \cdot \bar{u}_c = g^{k} g^{m + (p-1) - k} = g^{m + (p-1)} \equiv g^{m} \pmod{p}$ so that $u \cdot \bar{u}_c \equiv c \pmod{p}$. This proves existence for all cases. 

We will now show uniqueness. Suppose there are two such numbers $\bar{u}_{c,1}$ and $\bar{u}_{c,2}$. Both have the property that $u \cdot \bar{u}_{c,1} \equiv c \pmod{p}$ and $u \cdot \bar{u}_{c,2} \equiv c \pmod{p}$. This means that $u \cdot \bar{u}_{c,1} - u \cdot \bar{u}_{c,2} \equiv 0 \pmod{p}$, which implies $u (\bar{u}_{c,1} - \bar{u}_{c,2}) \equiv 0 \pmod{p}$. By the definition of modulus, we know that $p | u (\bar{u}_{c,1} - \bar{u}_{c,2})$. However, since $u$ and $p$ are coprime since $p$ is a prime, we know that $p \nmid u$. This means that $p | (\bar{u}_{c,1} - \bar{u}_{c,2}) $. However, this implies that $\bar{u}_{c,1} \equiv \bar{u}_{c,2} \pmod{p}$, which shows uniqueness. 
\end{proof}

Now that we have shown there exists a unique integer$\bar{u}_c$ modulo $p$  such that $u \cdot \bar{u}_c = c$, we know there is probability $1/p$ of selecting that integer uniformly at random from $\mathrm{Z}_p$. This proves the base case for the induction. Next, we will assume that our hypothesis holds up to vectors of length $n$. We must show it holds for vectors of length $n+1$. 

Consider the vector $\vec{u}$ of size $n+1$. It consists of a first element $v_1$, and another $n$ elements which can be thought of as a vector $\vec{u}_n$. The probability that a vector $\vec{v}$ chosen uniformly at random has $\vec{u} \cdot \vec{v} = c$ is the probability that $v_1 \cdot u_1 + \vec{u}_n \cdot \vec{v}_n = c$. Now notice that if $\vec{u}_n \cdot \vec{v}_n \equiv a \pmod{p}$, then we must have $v_1 \cdot u_1 \equiv c - a \pmod{p}$. It is clear that there is exactly one such element $c - a \in \mathrm{Z}_p$, which can be chosen with probability $1/p$. Therefore, the probability that $\vec{u} \cdot \vec{v} = c$ is given by:
\begin{eqnarray}
\sum_{i=0}^{p-1} \left( \frac{1}{p} \right)  \left( \frac{1}{p} \right) = \sum_{i=0}^{p-1} \left( \frac{1}{p} \right)^2 = \frac{1}{p}  
\end{eqnarray} 

The equation follows because there are $p$ possibilities for the result of $\vec{u}_n \cdot \vec{v}_n = a$, namely $\{0,1,\ldots, p-1\}$. Each of these have a probability of $1/p$ of occuring by our inductive hypothesis, and $v_1 \cdot v_2 = c - a$ also has a probability $1/p$ of occuring. This shows that the probability that $\vec{u} \cdot \vec{v} = c$ is $1/p$, which proves the theorem for arbitrary $n$.
\end{sol}

\subsection{Problem B}

\begin{prob}
You are given two vectors $\vec{x}, \vec{y} \in \mathrm{Z}_p^n$ such that $\vec{x} \neq \vec{y}$. Using the result from part (a), show that if $\vec{v}$ is a random vector as before, then $P[\vec{v} \cdot \vec{x} = \vec{v} \cdot \vec{y}] = 1/p$. 
\end{prob}

\begin{sol}
Since $v$ is a random vector, we know that the probability that $\vec{v} \cdot \vec{x} = c_x$ is $1/p$ for all $c_x \in \{0,1,\ldots, p-1\}$. Moreover, the probability that $\vec{v} \cdot \vec{y} = c_y$ is $1/p$ for all $c_y \in \{0,1,\ldots, p-1\}$. Now, we would like to find the probability that $c_x = c_y$. This is just the probability across $i$ in the set $\{0,1,2,\ldots, p-1\}$ that both $c_x$ and $c_y$ are equal to $i$. This is given by:
\begin{eqnarray}
\sum_{i=0}^{p-1} \left( \frac{1}{p} \right)^2 = \frac{1}{p}
\end{eqnarray}

This follows from problem (a) because there is a $1/p$ probability that $c_x = i$ and also a $1/p$ probability that $c_y = i$. Since $c_x$ and $c_y$ are independent, we can multiply probabilities, and sum across all $i \in \{0,1,\ldots, p-1 \}$. This shows that $P[\vec{v} \cdot \vec{x} = \vec{v} \cdot \vec{y}] = 1/p$. 
\end{sol}

\subsection{Problem C}

\begin{prob}
Using the result from part (b), show that if $A$ is an $m \times n$ matrix with each element chosen independently at random from $\mathrm{Z}_p$, then $P[A \vec{x} = A \vec{y}] = 1/ p^m$. 
\end{prob}

\begin{sol}
First, let us denote $A = [\vec{A}_1, \vec{A}_2, \ldots, \vec{A}_m]^{T}$ as a column vector consisting of $\vec{A}_i$, which is the $i$th row in $A$. We know from linear algebra that multiplication of matrices can simply be thought of as multiplication of individual row vectors in $A$ with $\vec{x}$. Thus, we know:
\begin{equation}
A \vec{x} = \left[ \begin{array}{c}
\vec{A}_1 \\
\vec{A}_2 \\
\vdots \\
\vec{A}_m 
\end{array}
\right] \vec{x} =  \left[ \begin{array}{c}
\vec{A}_1 \cdot \vec{x} \\
\vec{A}_2 \cdot \vec{x} \\
\vdots \\
\vec{A}_m \cdot \vec{x}
\end{array}
\right] 
\end{equation} 

Thus, in order for $A \cdot \vec{x} = A \cdot \vec{y}$, we must have $\vec{A}_i \cdot \vec{x} = \vec{A}_i \cdot \vec{y}$ for all $i \in \{1,2, \ldots, m\}$. We know from before that $P[\vec{A}_i \cdot \vec{x} = \vec{A}_i \cdot \vec{y}] = 1/p$, since $\vec{A}_i$ are all vectors chosen uniformly at random. Moreover, since we know that the vectors $\vec{A}_i$s are chosen independently by assumption, we can multiply the probabilities together. Since there are $m$ vectors that must be equal, we find that $P[A \vec{x} = A \vec{y}] = (1/p)^m = (1/p^m)$.
\end{sol}

\subsection{Problem D}

\begin{prob}
Conclude that the family $\mathcal{H}$ of all such functions $h_A(\vec{x}) = A \vec{x}$ where $A$ is an $m \times n$ matrix with elements in $\mathrm{Z}_p$ is universal.
\end{prob}

\begin{sol}
In order for a hash family $\mathcal{H}$ to be universal, we must have for $\vec{x} \neq \vec{y}$ that $P[h_A(\vec{x}) = h_A(\vec{y})] < 1/n$ for all $h \in \mathcal{H}$ where $n$ is the number of different possible vectors one can hash to. In our case, we know that the result of $A\vec{x}$ has $m$ rows, each with $p$ different possibilities. Thus, there are a total of $mp$ different possible vectors in our hash function's range. Therefore, we must show that $P[h_A(\vec{x}) = h_A(\vec{y})] < 1/(mp)$. However, we know from part (c) that $P[h_A(\vec{x}) = h_A(\vec{y})] = P[A \vec{x} = A \vec{y} ] = 1/p^m$. Thus, we must have $1/p^m < 1/(mp)$. This occurs when $p^m > mp$, or equivalently, when $p^{m-1} > m$.  

We note that if $p =2$, then $m > 4$ will result in $p^{m-1} = 2^3 = 8 > 4$. This is because $m = O(p^{m-1})$ and that $\lim_{m \to \infty} \frac{m}{p^{m-1}} = 0$. Since $p^{m-1}$ is an increasing function in $p$, we know that $p^{m-1} > m$ for all $p \geq 2$ if $m > 4$. Therefore, it is only necessary to choose $m \geq 4$ and any prime $p \geq 2$ in order for $\mathcal{H}$ to be a universal hash family.
\end{sol}

\subsection{Problem E}

\begin{prob}
Using the resut from part (a), devise a randomized algorithm to determine if $C = B \cdot A$. Show that your algorithm is correct with probability at least 90\%.
\end{prob}

\begin{sol}
First, we know that $A = (m \times n)$, $B = (k \times m)$, and $C = (k \times n)$ are the dimensions of the matrices. Knowing this, we can derive an algorithm as follows. We will create a random vector $\vec{x}$ of dimension $n$ by selecting $x_i \in \mathrm{Z}_2$ uniformly at random and independently. Then, we will compute $B \cdot A \cdot \vec{x}$ by first computing $A \cdot \vec{x}$, then computing $B \cdot (A \cdot \vec{x})$. We will also compute $C \cdot \vec{x}$. If $B \cdot A \cdot \vec{x} \neq C \cdot \vec{x}$, then return false. Otherwise, continue this procedure four times. If the result passes all four times, return true. 

First, we will analyze running time. Let us first analyze a single loop through the algorithm. Generating a random $n \times 1$ vector requires $O(n)$ time. Multiplying $A$, which is an $(m \times n)$ matrix, by $\vec{x}$ requires $O(mn)$ time, because each dot product requires $n$ multiplications and additions, and we must do this $m$ times. Thus, $A \cdot \vec{x}$ results in an $(m \times 1)$ sized vector. Computing $B \cdot (A \cdot \vec{x})$ will therefore require $O(km)$ time, since $B$ is a $(k \times m)$ sized matrix. Multiplying $B$ with an $(m \times 1)$ sized vector requires $m$ multiplications and additions for each entry in the result. Since there are $k$ entires, the running time is $O(km)$. Next, computing $C \cdot \vec{x}$ requires $O(kn)$ time because $C$ is a $(k \times n)$ matrix. Checking $B \cdot A \cdot \vec{x} = C \cdot \vec{x}$ requires $O(k)$ time because both the right hand side and left hand side are $(k \times 1)$ sized vectors. The total running time per loop is therefore $O(n + mn + km + kn + k) = O(mn + km + kn)$. Going through this loop at maximum four times only adds a constant factor to the run time. Therefore, the algorithm runs in $O(mn + km + kn)$ time. Thus, if $C$ is an $(n \times n)$ matrix, this algorithm runs in $O(n^2)$ time, which is very fast.

Now we shall show that the algorithm is correct at least 90\% of the time. First, if $B \cdot A = C$, then we see the algorithm will always be correct. This is because $B \cdot A \cdot \vec{x}$ will always be equal to $C \cdot \vec{x}$ in this case, and therefore the algorithm will go through four loops and return true. Next, if $B \cdot A \neq C$, we will show that the algorithm is correct at least 90\% of the time, which will show it is always correct at least 90\% of the time.

Let $D = B \cdot A$ and assume that $D \neq C$. We know from part $B$ that $P[ \vec{x} \cdot \vec{u} = \vec{x} \cdot \vec{v}] \leq 1/p$ where $\vec{x}$ is a random vector and $\vec{u} \neq \vec{v}$. Then the probability that the $i$th element in the output vector of $D \cdot \vec{x}$ differs from the $i$th element in the output vector of $C \cdot \vec{x}$ is $1/p$. The probability that any element differs is therefore $1 - (1 - 1/p)^k = 1 - (\frac{p-1}{p})^k$.  Since $p = 2$, we know the probability that any element differs, and thus that the algorithm is incorrect, is given by $1 - (\frac{1}{2})^k$. Since $k$ is the dimension of a matrix, and we know that $k \geq 1$, we know that this probability is bounded by $\frac{1}{2}$. The probability that we run the algorithm 4 times, and it says that $D \cdot \vec{x} = C \cdot \vec{x}$ for all four randomized vectors is bounded above by $(\frac{1}{2})^4 = 1 /16$. Thus, the probability that the algorithm returns the correct answer is bounded below by $1 - (\frac{1}{2})^4 = 15/16 \approx 0.93$. This shows that the algorithm is correct at least 90\% of the time.

Finally, we shall show an example of the algorithm at work. Let us pick the following matrices:
\begin{equation}
B = \left[ \begin{array}{c c}
1 & 0 \\
1 & 0 \\
0 & 1 
\end{array} \right], 
\hspace{0.5cm} A = \left[ \begin{array}{c c c}
1 & 0 \\
0 & 1 \\
\end{array} \right], 
\hspace{0.5cm} C = \left[ \begin{array}{c c}
1 & 0 \\
1 & 0 \\
1 & 1 
\end{array} \right]
\end{equation} 

We see immediately that $B \cdot A \neq C$ because the bottom left element of $B \cdot A$ is $0$, whereas the bottom left element of $C$ is 1. Let us run our algorithm to check this. First, we pick a random $n \times 1$ vector, where $n = 2$ in this case. Let us pick $\vec{x} = [0, 1]$. Then we first find $A \cdot \vec{x}$:
\begin{equation}
A \cdot \vec{x} = \left[ \begin{array}{c c c}
1 & 0 \\
0 & 1 \\
\end{array} \right] \cdot 
\left[ \begin{array}{c}
0 \\
1 
\end{array} \right] = 
\left[ \begin{array}{c}
0 \\
1 
\end{array} \right]
\end{equation}

Then we compute $B \cdot (A \cdot \vec{x})$:
\begin{equation}
B \cdot (A \cdot \vec{x}) = \left[ \begin{array}{c c}
1 & 0 \\
1 & 0 \\
0 & 1 
\end{array} \right] \cdot
\left[ \begin{array}{c}
0 \\
1 
\end{array} \right] =
\left[ \begin{array}{c}
0 \\ 
0 \\
1 
\end{array} \right]
\end{equation}

Now we copmare this result of $C \cdot \vec{x}$:
\begin{equation}
 C \cdot \vec{x} = \left[ \begin{array}{c c}
1 & 0 \\
1 & 0 \\
1 & 1 
\end{array} \right] \cdot 
\left[ \begin{array}{c}
0 \\
1 
\end{array} \right] =
\left[ \begin{array}{c}
0 \\
0 \\
1
\end{array} \right]
\end{equation}

The algorithm passes the first test. Therefore, we go onto the second loop. We pick a new random vector $\vec{x} = [1,1]$ and check if $B \cdot (A \cdot \vec{x}) = C \cdot \vec{x}$ again. This time, we find that:
\begin{equation}
B \cdot (A \cdot \vec{x})  =  \left[ \begin{array}{c c}
1 & 0 \\
1 & 0 \\
0 & 1 
\end{array} \right] \cdot 
\left[ \begin{array}{c c c}
1 & 0 \\
0 & 1 \\
\end{array} \right] \cdot 
\left[ \begin{array}{c}
1 \\
1
\end{array} \right] =
\left[ \begin{array}{c}
1 \\
1 \\ 
1 
\end{array} \right]
\end{equation}

\begin{equation}
C \cdot \vec{x} =  \left[ \begin{array}{c c}
1 & 0 \\
1 & 0 \\
1 & 1 
\end{array} \right] \cdot 
 \left[ \begin{array}{c}
1 \\
1 
\end{array} \right] \cdot 
 \left[ \begin{array}{c}
1 \\
1 \\
2 
\end{array} \right] 
\end{equation}

Since $B \cdot (A \cdot \vec{x}) \neq C \cdot \vec{x}$, the algorithm returns false, and it would be correct in this case.
\end{sol}

\end{document}